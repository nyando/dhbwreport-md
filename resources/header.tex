\usepackage{lipsum}
\usepackage{framed}
\usepackage{setspace}

% define default code listing style
\lstdefinestyle{dhpaperdefault}{
    basicstyle=\linespread{1.0}\normalsize\ttfamily,
    keepspaces=false,
    keywordstyle=\color{red},
    identifierstyle=\color{blue},
    commentstyle=\color{green},
    stringstyle=\color{orange},
    columns=flexible
}

% title page content
\makeatletter
\newcommand*{\projecttitle}[1]{\gdef\@projecttitle{#1}}
\newcommand*{\@projecttitle}{Titel}
\newcommand*{\projecttype}[1]{\gdef\@projecttype{#1}}
\newcommand*{\@projecttype}{Projektarbeit}
\newcommand*{\projectauthor}[1]{\gdef\@projectauthor{#1}}
\newcommand*{\@projectauthor}{Vorname Nachname}
\newcommand*{\studentid}[1]{\gdef\@studentid{#1}}
\newcommand*{\@studentid}{Matrikelnummer}

\newcommand*{\companyname}[1]{\gdef\@companyname{#1}}
\newcommand*{\@companyname}{Firmenname}
\newcommand*{\companylocation}[1]{\gdef\@companylocation{#1}}
\newcommand*{\@companylocation}{Firmensitz}
\newcommand*{\companylogo}[1]{\gdef\@companylogo{#1}}
\newcommand*{\@companylogo}{lion}
\newcommand*{\companyadvisor}[1]{\gdef\@companyadvisor{#1}}
\newcommand*{\@companyadvisor}{Vorname Nachname}

\newcommand*{\schoolname}[1]{\gdef\@schoolname{#1}}
\newcommand*{\@schoolname}{Dualen Hochschule Baden-Württemberg Karlsruhe}
\newcommand*{\schoollogo}[1]{\gdef\@schoollogo{#1}}
\newcommand*{\@schoollogo}{dhbw-logo}
\newcommand*{\schooladvisor}[1]{\gdef\@schooladvisor{#1}}
\newcommand*{\@schooladvisor}{Vorname Nachname}

\newcommand*{\deadline}[1]{\gdef\@deadline{#1}}
\newcommand*{\@deadline}{01.01.1970}
\newcommand*{\duration}[1]{\gdef\@duration{#1}}
\newcommand*{\@duration}{1000 Jahre}

\newcommand*{\degreetype}[1]{\gdef\@degreetype{#1}}
\newcommand*{\@degreetype}{Professor of Science}
\newcommand*{\subject}[1]{\gdef\@subject{#1}}
\newcommand*{\@subject}{Universalwissenschaften}
\newcommand*{\courseid}[1]{\gdef\@courseid{#1}}
\newcommand*{\@courseid}{WTF19BBQ}

% redefine maketitle command to create a neato title page
\newcommand*{\maketitlepage}{
    \begin{titlepage}
        \begin{center}
            \vspace*{-2cm}
            {\includegraphics[width=4cm]{\@companylogo}\hfill\includegraphics[width=4cm]{\@schoollogo}} \\
            \vspace*{2cm}
            {\Huge \@projecttitle} \\
            \vspace*{1cm}
            {\Huge\scshape \@projecttype} \\
            \vspace*{1cm}
            {\large für die Prüfung zum} \\
            \vspace*{0.5cm}
            {\Large \@degreetype} \\
            \vspace*{0.5cm}
            {\large des Studienganges \@subject} \\
            \vspace*{0.5cm}
            {\large an der} \\
            \vspace*{0.5cm}
            {\large \@schoolname} \\
            \vspace*{0.5cm}
            {\large von} \\
            \vspace*{0.5cm}
            {\large\bfseries \@projectauthor} \\
            \vspace*{1cm}
            {\large Abgabedatum \@deadline} \\
            \vfill
        \end{center}
        \begin{tabular}{l@{\hspace{2cm}}l}
            Bearbeitungszeitraum          & \@duration         \\
            Matrikelnummer                & \@studentid        \\
            Kurs                          & \@courseid         \\
            Ausbildungsfirma              & \@companyname      \\
                                          & \@companylocation  \\
            Betreuer der Ausbildungsfirma & \@companyadvisor   \\
            Gutachter der Studienakademie & \@schooladvisor      \\
        \end{tabular}
    \end{titlepage}
}

% redefine maketitle command to create a neato title page
\newcommand*{\nocompanytitlepage}{
    \begin{titlepage}
        \begin{center}
            \vspace*{-2cm}
            {\hfill\includegraphics[width=4cm]{\@schoollogo}} \\
            \vspace*{2cm}
            {\Huge \@projecttitle} \\
            \vspace*{1cm}
            {\Huge\scshape \@projecttype} \\
            \vspace*{1cm}
            {\large für die Prüfung zum} \\
            \vspace*{0.5cm}
            {\Large \@degreetype} \\
            \vspace*{0.5cm}
            {\large des Studienganges \@subject} \\
            \vspace*{0.5cm}
            {\large an der} \\
            \vspace*{0.5cm}
            {\large \@schoolname} \\
            \vspace*{0.5cm}
            {\large von} \\
            \vspace*{0.5cm}
            {\large\bfseries \@projectauthor} \\
            \vspace*{1cm}
            {\large Abgabedatum \@deadline} \\
            \vfill
        \end{center}
        \begin{tabular}{l@{\hspace{2cm}}l}
            Bearbeitungszeitraum          & \@duration         \\
            Matrikelnummer                & \@studentid        \\
            Kurs                          & \@courseid         \\
            Gutachter der Studienakademie & \@schooladvisor      \\
        \end{tabular}
    \end{titlepage}
}

% statement of originality
\newcommand*{\makestatement}{
    \newpage
    \thispagestyle{empty}
    \begin{framed}
    \begin{center}
    \Large\bfseries Erklärung
    \end{center}
    \medskip
    \noindent
    Ich versichere hiermit, dass ich meine \@projecttype\ mit dem Thema:
    \enquote{\@projecttitle}
    selbstständig verfasst und keine anderen als die angegebenen Quellen und Hilfsmittel benutzt habe.
    Ich versichere zudem, dass die eingereichte elektronische Fassung mit der gedruckten Fassung übereinstimmt.

    \vspace{3cm}
    \noindent
    \underline{\hspace{4cm}}\hfill\underline{\hspace{6cm}}\\
    Ort~~~~~Datum\hfill Unterschrift\hspace{4cm}
    \end{framed}
}

% confidentiality statement
\newcommand*{\makenda}{
    \vfill
    \begin{framed}
    \begin{center}
    \Large\bfseries Sperrvermerk
    \end{center}
    \medskip
    \noindent
    Der Inhalt dieser Arbeit darf weder als Ganzes noch in Auszügen Personen außerhalb des Prüfungsprozesses und des Evaluationsverfahrens zugänglich gemacht werden, sofern keine anderslautende Genehmigung vom Dualen Partner vorliegt.
    \end{framed}
}
